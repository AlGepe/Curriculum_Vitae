%% start of file `template.tex'.
%% Copyright 2006-2013 Xavier Danaux (xdanaux@gmail.com).
%
% This work may be distributed and/or modified under the
% conditions of the LaTeX Project Public License version 1.3c,
% available at http://www.latex-project.org/lppl/.


\documentclass[11pt,a4paper,sans]{moderncv}        % possible options include font size ('10pt', '11pt' and '12pt'), paper size ('a4paper', 'letterpaper', 'a5paper', 'legalpaper', 'executivepaper' and 'landscape') and font family ('sans' and 'roman')

\usepackage[square, numbers, comma, sort&compress]{natbib}  % Use the "Natbib" style for the references in the Bibliography
\usepackage{verbatim}  % Needed for the "comment" environment to make LaTeX comments
\usepackage{float} 
\usepackage{selinput}
\usepackage{scrextend}
% moderncv themes
\moderncvstyle{classic}                             % style options are 'casual' (default), 'classic', 'oldstyle' and 'banking'
\moderncvcolor{blue}                               % color options 'blue' (default), 'orange', 'green', 'red', 'purple', 'grey' and 'black'
%\renewcommand{\familydefault}{\rmdefault}         % to set the default font; use '\sfdefault' for the default sans serif font, '\rmdefault' for the default roman one, or any tex font name
%\nopagenumbers{}                                  % uncomment to suppress automatic page numbering for CVs longer than one page
\newcommand\blfootnote[1]{%
  \begingroup
  \renewcommand\thefootnote{}\footnote{#1}%
  \addtocounter{footnote}{-1}%
  \endgroup
}

%\usepackage[spanish]{babel}
%\selectlanguage{spanish}
% character encoding
%\usepackage[utf8]{inputenc}                       % if you are not using xelatex ou lualatex, replace by the encoding you are using
%\usepackage{CJKutf8}                              % if you need to use CJK to typeset your resume in Chinese, Japanese or Korean
% adjust the page margins
\usepackage[scale=0.75]{geometry}
%\usepackage{hyperref}
\usepackage{fancyhdr}
\pagestyle{fancy} % All pages have headers and footers
\fancyhead{} % Blank out the default header
\fancyfoot{} % Blank out the default footer
\pagestyle{fancy} % All pages have headers and footers
\SelectInputMappings{%
	aacute={á},
	ntilde={ñ},
	Euro={€}
}

%\renewcommand{\footnote}{\arabic{footnote}}
%\setlength{\hintscolumnwidth}{3cm}                % if you want to change the width of the column with the dates
%\setlength{\makecvtitlenamewidth}{10cm}           % for the 'classic' style, if you want to force the width allocated to your name and avoid line breaks. be careful though, the length is normally calculated to avoid any overlap with your personal info; use this at your own typographical risks...

% personal data
\name{Alvaro}{Diez}
\title{Curriculum Vitae}                               % optional, remove / comment the line if not wanted
\address{Aleje Jerozolimskie 196A 2}{02-486 Warszawa}{Poland}% optional, remove / comment the line if not wanted; the "postcode city" and and "country" arguments can be omitted or provided empty
\phone[mobile]{+34 690 603 466}                   % optional, remove / comment the line if not wanted
%			\phone[fixed]{+34~942~219~639}                    % optional, remove / comment the line if not wanted
%\phone[fax]{+3~(456)~789~012}                      % optional, remove / comment the line if not wanted
\email{alvarodiezgp@gmail.com}                               % optional, remove / comment the line if not wanted
\homepage{github.com/AlGepe}%\href{https://www.github.com/AlGepe}{github.com/AlGepe}}                               % optional, remove / comment the line if not wanted
%\homepage{www.johndoe.com}                         % optional, remove / comment the line if not wanted
%\extrainfo{additional information}                 % optional, remove / comment the line if not wanted
%\photo[64pt][0.4pt]{youpp}                       % optional, remove / comment the line if not wanted; '64pt' is the height the picture must be resized to, 0.4pt is the thickness of the frame around it (put it to 0pt for no frame) and 'picture' is the name of the picture file
%\quote{Some quote}                                 % optional, remove / comment the line if not wanted

% to show numerical labels in the bibliography (default is to show no labels); only useful if you make citations in your resume
%\makeatletter
%\renewcommand*{\bibliographyitemlabel}{\@biblabel{\arabic{enumiv}}}
%\makeatother
%\renewcommand*{\bibliographyitemlabel}{[\arabic{enumiv}]}% CONSIDER REPLACING THE ABOVE BY THIS

% bibliography with multiply entries
%\usepackage{multibib}
%\newcites{book,misc}{{Books},{Others}}
%----------------------------------------------------------------------------------
%            content
%----------------------------------------------------------------------------------
\begin{document}
%\begin{CJK*}{UTF8}{gbsn}                          % to typeset your resume in Chinese using CJK
%-----       resume       ---------------------------------------------------------
\makecvtitle

\iffalse
	\section{Pre - University Studies}
	\cventry{2008-2010}{Bachillerato Español}{Instituto Santa Clara}{Santander}{\textit{High School Studies}}{}  % arguments 3 to 6 can be left empty
	\begin{itemize}
		\item{Final grade: 8.1/10 Including selectividad\footnote{National university entry exam} }
			\newline
	\end{itemize}
	\cventry{2008-2010}{International Baccalaureate}{Instituto Santa Clara}{Santander}{\textit{High School Studies}}{Studied simultaneously with "Bachillerato" and completed in two years following High School's schedule}
	\begin{itemize}
		\item Final grade: 31 pt
			\begin{itemize} 
				\item Including 6/7 in Physics
			\end{itemize}
	\end{itemize}
\fi

\section{University Studies}
\cventry{2010-2017}{Degree in Physics}{Universidad de Cantabria}{Santander}{\textit{Final Grade}}{7.45/10}

\cventry{2017-Current}{Masters in Physics}{Specialisation in "Mathematical and computer modeling of physical processes"}{University of Warsaw}{Warsaw}{}

\textit{\textbf{Remarks:}}
\begin{itemize}%
	\item \textit{2012-2013} Erasmus in Bergen (Norway)
	\item Bachelor's thesis: \newline{}
		\cvitem{Title}{\emph{Fast simulation of transients in irradiated silicon detectors}}
		\cvitem{Supervisor}{Marcos Fernández García (UC-CERN)}
		% \cvitem{Published Paper}{}

\end{itemize}

%\section{Additional Courses}
%\cventry{February 2017-Current}{Introduction to Machine Learning}{\href{https://www.udacity.com/course/intro-to-machine-learning--ud120}{Udacity ud120}}{}{}{}{}{}
%\cventry{February 2017-Current}{Introduction to Parallel Programing (CUDA)}{\href{https://www.udacity.com/course/intro-to-parallel-programming--cs344}{Udacity cs344}}{}{}{}{}{}


\section{Experience}

\subsection{Laboral} % (fold)
\label{sub:laboral}
\cventry{June-September (2015)}{Internship at CERN}{CERN, Geneva}{Member of the PH-DD-SSD group}{\newline Main developer of \href{https://github.com/IFCA-HEP/TRACS}{TRACS}, a fast simulator for irradiated silicon detectors in C++}{Summer Student Program}{}

\cventry{February-October (2016)}{Internship at CERN}{CERN, Geneva}{Member of the EP-DT-FS-Gas group}{\newline Controlling gas mixtures inside selected gaseous detectors from the LHC and providing technical assistance to researchers in gas-related matters}{Technical Student Program}{}

% subsection laboral (end)
%\newpage
\subsection{Vocational}
\cventry{2013-2017}{Founder and member of the Physics Mentor Group}{Santander}{}{\newline Group founded by 4 students to help new university students and publicise science}{}%The Physics Mentor Group also:%
%\begin{itemize}%
	%\item Helped promoting physics amongst high school students during doors open days
	%\item Took part in the European researcher's night in Santander (2014 \& 2015)
	%\item Took part in the first science fair in Cantabria [December 2015]
	%\item Organised the Gran Gymnkhana de Ciencias \newline (the first skill-based contest for students ever organised in the UC)
	%\item Co-organized the Gymnkhana de la Luz \newline (a skill-based contest for students of all ages for the International Year of Light)
		%% \begin{itemize}%
		%% \item Sub-achievement (a);
		%% \item Sub-achievement (b), with sub-sub-achievements (don't do this!);
		%%   \begin{itemize}
		%%   \item Sub-sub-achievement i;
		%%   \item Sub-sub-achievement ii;
		%%   \item Sub-sub-achievement iii;
		%%   \end{itemize}
		%% \item Sub-achievement (c);
		%% \end{itemize}
		%%\item Achievement 3.
%\end{itemize}
% \hspace{5pt}
% \newline{}
\subsection{Journalism}
\cventry{2012--2016}{Blog Journalist}{}{}{}{}
\vspace{-15pt}
	\begin{addmargin}[6.5em]{0em}% 1em left, 2em right
	\item[] - \underline{\textit{El Androide Libre}} One year as journalist for one of the most visisted Android weblogs in Spanish
	\item - \underline{\textit{Medciencia}} Two years as science journalist for Medciencia
	\end{itemize}
\end{addmargin}% 1em left, 2em right
{}{}

\subsection{Sports}
\cventry{2007--2017}{Handball Referee}{8 years of handball refereeing}{\newline Highest Category: National Referee (2011-2014)}{\newline Refereed several National and International Championships u-18}{}
%\textit{\textbf{Achievements:}}
%\begin{itemize}
	%\item Became national referee in 2010
	%\item Member of the Spanish Young referees Handball Project 2011-2012%\footnote{Project ended in 2012 after the new president of the Spanish Handball Federation was elected}
	%\item Designated referee for the international tournament "Granollers Cup" 2010 \& 2011
	%\item One year refereeing in Norway up to third level \textit{During the Erasmus}
	%\item Several nominations as referee for Spanish Championships up to under 18 y.o. level
%\end{itemize}
\cventry{2009--2010}{Handball coach of a kids team}{}{}{}{}

% \cventry{2014--2016}{Science Blog Journalist}{Medciencia}{}{}{Currently working as physics journalist for the Medciencia weblog}
%General Achievements:
%\begin{itemize}
	%\item Three times featured in the main page of Meneame.com\footnote{Top site of news in Spain (community based)}
%\end{itemize}
\section{Languages}
\cvitemwithcomment{English}{C1 Level}{Cambridge Certificate in Advance English obtained in May 2010}
\cvitemwithcomment{French}{Basic Level}{Studied french at school 2004-2008 \& lived in France for 11 months}

\section{Computer skills}
\cventry{Software Development}{}{}{}{}{}%C++/C, Java, gdb, UML, Matlab, ROOT}{Basic-Intermediate Level}{Experience developing software (mainly physics simulators and physics oriented software) in the three aforementioned languages and UML diagrams.}{Experience using $gdb$ ( and $ddd$) debugger}{Basic knowledge of HTML, CSS, JavaScript, Python, bash scripting, FORTRAN, AWK, SED, and Cuda.}
\vspace{-35pt}
	\begin{addmargin}[6.5em]{0em}% 1em left, 2em right
		\item[] \textbf{C/C++, Python, git}
		\item  Intermediate knowledge with experience in data analysis
		\item[] \textbf{Java, gdb/ddd, bash, Matlab, UML, ROOT, Cuda, MPI, OpenMP}
		\item  Basic knowledge with experience
	\end{itemize}
\end{addmargin}% 1em left, 2em right

\vspace{10pt}
\cventry{Operative Systems}{}{}{}{}{}%C++/C, Java, gdb, UML, Matlab, ROOT}{Basic-Intermediate Level}{Experience developing software (mainly physics simulators and physics oriented software) in the three aforementioned languages and UML diagrams.}{Experience using $gdb$ ( and $ddd$) debugger}{Basic knowledge of HTML, CSS, JavaScript, Python, bash scripting, FORTRAN, AWK, SED, and Cuda.}
\vspace{-35pt}
	\begin{addmargin}[6.5em]{0em}% 1em left, 2em right
		\item[] \textbf{Linux, OSX, Windows}
		\item  Intermediate/Advance and fluent usage of command line interface as well as remote work (e.g.: ssh, sshfs...)
	\end{itemize}
\end{addmargin}% 1em left, 2em right

\vspace{10pt}
\cventry{Office Suites}{}{}{}{}{}
\vspace{-30pt}
	\begin{addmargin}[6.5em]{0em}% 1em left, 2em right
		\item[] \textbf{MS Office, LibreOffice, \LaTeX}
		\item  IIntermediate level with experience mainly in Latex and Excel
	\end{itemize}
\end{addmargin}% 1em left, 2em right

\vspace{10pt}
\cventry{Text Editors}{}{}{}{}{}
\vspace{-30pt}
	\begin{addmargin}[6.5em]{0em}% 1em left, 2em right
		\item[] \textbf{Vim}
		\item  Intermediate/Advance with extended experience as main editor including IDE-like plugins
	\end{itemize}
\end{addmargin}% 1em left, 2em right
%\cvitem{category 3}{XXX, YYY, ZZZ}

\section{Additional Information}
\cvitem{}{Driving License \textit{Category: B}}
% \cvitem{hobby 2}{Description}
% %\cvitem{hobby 3}{Description}

\newpage
\center{\huge Links to relevant information}
\vspace{2cm}
\section{References and Publications}
\subsection{Academic} % (fold)
\label{sub:}

\cvlistitem{\href{https://github.com/AlGepe}{Personal GitHub} (https://github.com/AlGepe)}
\cvlistitem{Publication at CERN \href{http://cds.cern.ch/record/2057142}{Developing a fast simulator for irradiated silicon detectors} (http://cds.cern.ch/record/2057142)}
\cvlistitem{\href{https://indico.cern.ch/event/447412/}{Final talk on project at CERN:} \href{https://docs.google.com/presentation/d/1ItaYaDn644FlFxUDNH7XW6icOPsbcM7SAt67wIu7BC8/edit#slide=id.p}{TRACS radiation upgrade v2.0}(https://indico.cern.ch/event/447412)}
\cvlistitem{Talk at Summer Student Sessions at CERN \href{http://cds.cern.ch/record/2042184}{Developing a fast simulator for irradiated silicon detectors}}
\cvlistitem{Public Gas Monitoring web: \href{http://test-gasmonitoring.web.cern.ch/#}{Test Gas Monitoring}}
\cvlistitem{Bachelor thesis:
	\begin{itemize}
		\item Personal repository(tex format): \href{(https://github.com/AlGepe/TFG)}{https://github.com/AlGepe/TFG }
	%\item University's repository(pdf): \href{}{}
\end{itemize}
} %CVLISTITEM BACHELOR THESIS
%\cvlistitem{}

\vspace{50pt}
\subsection{Journalism (Hyperlinked)} % (fold)
{\underline{ Most succesful articles. \newline \hspace{4pt}}
\label{sub:journalism}
\cvlistitem{\href{http://www.elandroidelibre.com/2013/02/supercondensadores-de-grafeno-baterias-que-se-cargaran-en-30-minutos-y-duran-un-dia-entero.html}{Graphene Supercapacitors: batteries will charge in 30 seconds and last all day}}
\cvlistitem{\href{www.medciencia.com/por-que-el-led-azul-merece-un-premio-nobel-en-fisica-y-el-led-verde-no/}{Why does blue LED deserve a Nobel Prize in Physics and green LED doesn't?}}
\cvlistitem{\href{http://www.medciencia.com/la-curiosa-historia-del-monstruo-invisible-que-destrozaba-vagones-en-siberia}{The story of the "invisible monster" that wrecked trains in Siberia}}
\cvlistitem{\href{http://www.medciencia.com/como-pensar-en-11-dimensiones-y-no-morir-en-el-intento/}{How to think in 11 dimensions and not die in the process}}
%\end{vplace}
\blfootnote{I hereby give consent for my personal data included in my application to be processed for the purposes of the recruitment process under the Personal Data Protection Act as of 29 August 1997, consolidated text: Journal of Laws 2016, item 922 as amended.}


\clearpage
%\vspace{8cm}
\ablfootnote{I hereby give consent for my personal data included in my application to be processed for the purposes of the recruitment process under the Personal Data Protection Act as of 29 August 1997, consolidated text: Journal of Laws 2016, item 922 as amended.}

\clearpage
\end{document}
